\documentclass{article}
\usepackage{graphicx}

\title{Longest Path in a Set of Files - MapReduce Implementation in C++}
\author{Your Name(s)}

\begin{document}
\maketitle

\section{Introduction}
This report presents the implementation of a MapReduce framework to find the longest file path in a distributed set of files, using C++. The files are distributed across multiple laptops, and each file contains one path per line. The task is to identify the longest file path by length.

\section{MapReduce Implementation}
The system consists of two main components: the Mapper and the Reducer. The Mapper processes each input line (representing a file path), calculates its length, and outputs a key-value pair. The Reducer receives these pairs, compares the lengths, and outputs the longest file path(s).

\subsection{Mapper}
The Mapper performs the following steps:
\begin{enumerate}
  \item Reads each line representing a file path.
  \item Computes the length of the file path.
  \item Emits a key-value pair, where the key is the length of the path, and the value is the file path itself.
\end{enumerate}

\subsection{Reducer}
The Reducer performs the following steps:
\begin{enumerate}
  \item Receives key-value pairs from the Mapper.
  \item Compares the path lengths and keeps track of the longest path(s).
  \item Outputs the longest path(s) after processing all input pairs.
\end{enumerate}

\section{Diagram of the Process}
\begin{figure}[h!]
    \centering
    \includegraphics[width=0.5\textwidth]{mapreduce_diagram.png}
    \caption{Flowchart of the MapReduce process for finding the longest path.}
\end{figure}

\section{Conclusion}
The C++ implementation of the MapReduce framework successfully identifies the longest file path by length in a distributed environment. This approach efficiently handles large datasets and provides an optimal solution to the problem.

\end{document}

